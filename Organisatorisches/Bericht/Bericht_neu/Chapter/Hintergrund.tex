\chapter{Hintergrund}

Für viele Besucher sind die Bayrischen Alpen ein beliebtes Gebiet, im Sommer sind Wanderer anzutreffen und im Winter vorrangig Skitouren- und Schneeschuhgeher. Jedoch stellt das Alpengebiet auch das Territorium von verschiedenen Tierarten dar. Es ist zu beobachten, dass es immer wieder zu räumlichen Konfliktsituation zwischen Menschen und Tieren kommt. Vor allem im Winter dringen die Sportler tief in die Rückzugsgebiete der Tiere ein und zwingen diese zur Flucht. Um die räumlichen Konflikte zu entschärfen, werden die Besuchenden vor Ort informiert und durch Lenkungsmaßnahmen gesteuert. Mit dem Ziel einen nachhaltigen Tourismus gewährleisten zu können wurde ein Aktionstag seitens des Alpenvereins initiiert. Der Aktionstag ist ein Bestandteil der Kampagne „Natürlich auf Tour“ und dient der Sensibilisierung von Besuchern für einen verantwortungsvollen Umgang mit der Natur. (Quelle: https://www.alpenverein.de/natur/aktionstag-natuerlich-auf-tour_aid_32844.html)
   Um das Besucherverhalten besser zu verstehen und beeinflussen zu können, werden die Signale der LVS-Geräts, das ein Skitourengeher in der Regel dicht am Körper trägt, ausgewertet und analysiert.Ein LVS-Gerät ist ein Lawinenverschüttungssuchgerät und gehört zur Ausrüstung von Sportlern die in Schneegebieten unterwegs sind. Mit Hilfe eines LVS-Geräts können Personen, die von einer Lawine verschüttet worden sind, geortet werden (Quelle: Semmel, C., and D. Stopper. "LVS Geräte im Test." DAV Panorama-Mitteilungen des Deutschen Alpenvereins, München, Germany 2.2007 (2007): 88-91). Die Untersuchung über die Mitnahme von LVS-Geräten wurde anhand von Checkpoints gemessen. An den LVS-Checkpoints kann der der Wintersportler sein Gerät auf dessen Funktionalität testen. Wenn eine Person ein eingeschaltetes LVS-Gerät bei sich trägt und an einem Checkpoint vorbei läuft, wird dies als ein LVS-Signal registriert und das Gerät leuchtet grün auf. Falls eine Person, die kein Gerät dabei hat, bzw. eins mit sich führt, dieses jedoch nicht eingeschaltet ist am Checkpoint vorbeiläuft, dann wird dies als eine Infrarot-Signal (abgekürzt: IR-Signal) identifiziert und das Messgerät leuchtet rot auf. Die dabei anonym erhobenen Daten werden für die weitere Untersuchung betrachtet. 
   In unserem Projekt betrachten wir nicht das ganze Alpengebiet als Untersuchungsort, sondern beschränken uns auf den Spitzingsee. 
   (Untertitel: Untersuchungsraum: Spitzingsee (gelb umrandet) mit dem Berg Taubenstein (links oben abgebildet))
(Quelle: https://www.bergpixel.de/spitzingsee-rotwand-taubenstein-3/)
Der Spitzingsee ist ein beliebter Anlaufpunkt für Skitouren- und Schneeschuhgeher, aber auch das Revier von Wildtieren und daher als Untersuchungsraum für das Projekt geeignet.  Am Taubenstein-Parkplatz sind zur Hochsaison zwischen 500 bis 1000 Besucher anzutreffen, weshalb dieser Punkt für das Aufstellen von zwei Checkpoints genutzt wurde. (Quelle: Panorama-6-2018-Skitouren-naturvertäglich-Bayern.pdf)
(Untertitel: LVS-Checkpoints an der Nord und Südseite des Taubenstein-Parkplatzes)
(Quelle: Roman Excel-Datei)
Es gibt insgesamt zwei Checkpoints die einmal auf der Nord- und Südseite des Parkplatzes kurz vor Aufstieg des Bergs aufgestellt sind.
   
Im Fokus der Analyse steht der Zusammenhang zwischen dem Anteil der Besucher mit LVS-Gerät an allen Besuchern und verschiedenen Umweltfaktoren. Die von den Checkpoints generierten Daten werden im nächsten Kapitel näher beschrieben. 
