\documentclass{beamer}

%Style
\usetheme[progressbar=frametitle]{metropolis}
\setbeamertemplate{frame numbering}[fraction]
\useoutertheme{metropolis}
\useinnertheme{metropolis}
\usefonttheme{metropolis}
\usecolortheme{spruce}
\setbeamercolor{background canvas}{bg=white}
%http://latexcolor.com/ um Farbe zu definieren 
\definecolor{lmugreen}{rgb}{.36,.54,.66}
\usecolortheme[named=lmugreen]{structure}


%Packages
\usepackage[utf8]{inputenc} 
\usepackage[ngerman]{babel} 
\usepackage{graphicx} 
\usepackage{booktabs} 
\usepackage{amsmath}
\usepackage{amssymb}



%Einstellungen der Praesi
\title{Projekt LVS-IR-Taubenstein}
\author{\small \textbf{Projektpartner}: Sascha Filimon, Roman Ossner\\ \textbf{Gruppenbetreuer}: Andr\'{e} Klima\\ 
\textbf{Projektgruppe}: Alexander Fogus, Lea Vanheyden, Zorana Spasojevic
}
\institute{Ludwig Maximilians Universität}

\logo{\includegraphics[width=.85cm]{lmu_logo.png}\hfill}

%Beginn Praesi
\begin{document}
\metroset{block=fill}
\begin{frame}
\titlepage
\end{frame}

\begin{frame}[t]{Inhaltsverzeichnis}
\begin{enumerate}
    \item Hintergrund
    \item Datengrundlage
    \item Aufgabenstellung
    \item Binäre Regressionsmodelle
    \item Probleme
\end{enumerate}   
\end{frame}

 
\begin{frame}[t]{1. Hintergrund}\vspace{4pt}
\begin{itemize}
	\item Konfliktsituation zwischen Mensch und Natur/Tierreich im Alpengebiet 
	\item Kooperation des Departments für Geographie an der LMU, Lawinencamp Bayern, Gebietsbetreuer Mangfallgebirge, Alpenregion Tegernsee/Schliersee und DAV Sektion München
	\item  Speziellen Untersuchungen am Spitzingsee (beliebte Gegend für Sportler und Wildtiere) 
	\item Wie verhalten sich die Besucher und wie kann man dieses Verhalten steuern? 
	\item Dazu Untersuchung über die Mitnahme von LVS-Geräten anhand von Checkpoints und manueller Datenerhebung
\end{itemize}
\end{frame}

\begin{frame}[t]{2. Datengrundlage}\vspace{4pt}
\vspace{1pt}
\begin{itemize}
	\item Untersuchungsgegenstand: Wintersportler (vorrangig Skitourengänger \& Schneeschuhgeher) 
	\item Untersuchungszeitraum: Wintersaison 18/19 \\
	\rotatebox[origin=c]{180}{$\Lsh$} Genauer Zeitraum: 21.12.2018 - 12.04.2019
	\item Checkpoints an zwei Routen (Nord- und Südseite) erfassen: \\
	\rotatebox[origin=c]{180}{$\Lsh$} Messungen insgesamt: 31574 \\
	\rotatebox[origin=c]{180}{$\Lsh$} Personen mit LVS-Gerät: 8468 \\
	\rotatebox[origin=c]{180}{$\Lsh$} Personen ohne LVS-Gerät: 23106
\end{itemize}	
\end{frame}

\begin{frame}{2. Datengrundlage}

	\begin{figure}[H]
    \centering
    \includegraphics[width=.85\textwidth]{plot_messerfassung_datum.png}
    %\caption{Messerfassungen von Mitte Dezember bis Mitte April}
    \end{figure}
    
\end{frame}

\begin{frame}[t]{2. Datengrundlage}\vspace{4pt}
\begin{itemize}
    \item Weitere Variablen: \\
    \rotatebox[origin=c]{180}{$\Lsh$} Datum \\
    \rotatebox[origin=c]{180}{$\Lsh$} Uhrzeit \\
    \rotatebox[origin=c]{180}{$\Lsh$} Tageslänge \\
    \rotatebox[origin=c]{180}{$\Lsh$} Temperatur \\
    \rotatebox[origin=c]{180}{$\Lsh$} Schneehöhe \\ \rotatebox[origin=c]{180}{$\Lsh$} Sonnenstrahlung \\
    \rotatebox[origin=c]{180}{$\Lsh$} Wochentag \\
    \rotatebox[origin=c]{180}{$\Lsh$} Feiertag \\
    \rotatebox[origin=c]{180}{$\Lsh$} Lawinenwarnstufe \\
    \item Durch manuelle Stichproben wurden die Messungen der Checkpoints als fehlerhaft erkannt
\end{itemize}

\end{frame}
    


\begin{frame}[t]{3. Aufgabenstellung}\vspace{4pt}
\begin{enumerate}
    \item\textbf{Modell:} Anteil der Skitourengänger mit LVS-Gerät in Abhängigkeit von anderen Faktoren (wie z.B. Uhrzeit, Temperatur, Schneehöhe)
    \item Einflussfaktoren von denen die Messfehler abhängen, welcher Art und Struktur
    \item Hypothese: Unter Berücksichtigung der Erkenntnisse über die Messfehler \\
    \rotatebox[origin=c]{180}{$\Lsh$} Wie beeinflussen die Messfehler die geschätzten Abhängigkeiten?
\end{enumerate}
    
\end{frame}

\begin{frame}[t]{4. Binäre Regressionsmodelle}\vspace{4pt}
\begin{block}{Daten}
Die binäre Zielvariablen $y_{i}$ sind $0/1$-kodiert und bei gegebenen Kovariablen $x_{i1},...,x_{ik}$ (bedingt) unabhängig.
\end{block}
\begin{block}{Modelle}
Die Wahrscheinlichkeit $\pi_{i}=P(y_{i}=1|x_{i1},...,x_{ik})$ und der lineare Prädiktor:
$$\eta _{i}= \beta _{0}+ \beta _{1}x_{i1}+...+ \beta  _{k}x_{ik}=x'_{i} \beta $$
sind durch eine Responsefunktion $h(\eta)\in[0,1]$ miteinander verknüft:
$$\pi_{i}=h(\eta_{i}).$$    
\end{block}
\tiny{(Quelle: Fahrmeir, Ludwig, et al. Regression; Prof. Dr. Helmut Küchenhoff Vorlesungsskript)}
\end{frame}

\begin{frame}[t]{4. Binäre Regressionsmodelle}\vspace{4pt}
\begin{block}{Logit-Modell:}
$$\pi=\frac{exp(\eta)}{1+exp(\eta)} \iff log\frac{\pi}{1-\pi}=\eta.$$
Interpretation: log odds durch lineares Modell beschreibbar.
\end{block}
\begin{block}{Probit-Modell:}
$$\pi= \Phi(\eta) \iff \Phi^{-1}(\pi)=\eta.$$
 Interpretation: z-transformierte Wahrscheinlichkeiten sind durch lineares Modell beschreibbar.\\
 Interpretation der $\beta$ durch marginale Effekte
\end{block}
\tiny{(Quelle: Fahrmeir, Ludwig, et al. Regression; Prof. Dr. Helmut Küchenhoff Vorlesungsskript )}
\end{frame}

\begin{frame}[t]{4. Binäre Regressionsmodelle}\vspace{4pt}
\begin{itemize}
\item Gruppiertes Logit- oder Probit-Modell würden sich eignen
\item Statistische Analysen mit Logit- und Probit-Modellen führen zu ähnlichen Resultaten 
\end{itemize} 

\end{frame}


\begin{frame}[t]{5. Probleme}\vspace{4pt}
\begin{enumerate}
    \item Modellwahl\\
    \rotatebox[origin=c]{180}{$\Lsh$} Longitudinal-Data-Analysis: z.B. Schneehöhe an Tag 1 korreliert mit Schneehöhe an Tag 2 \\
    \rotatebox[origin=c]{180}{$\Lsh$} Problem: Daten über gewissen Zeitraum gemessen, an manchen Tagen keine Messungen
    \item Modellvergleich\\
    \rotatebox[origin=c]{180}{$\Lsh$} Logit- oder Probit-Modell \\
    \rotatebox[origin=c]{180}{$\Lsh$} Variablenselektion
    \item Überdispersion
\end{enumerate}
\end{frame}

\begin{frame}[t]{5.Probleme}\vspace{4pt}
\begin{block}{Überdispersion:}
Für gruppierte Daten lässt sich die Varianz innerhalb der Gruppe abschätzen durch $\frac{\bar y_{i}(1- \bar y_{i})}{n_{i}}$, da $\bar y_{i}$ der ML-Schätzer für $\pi_{i}$ basierend auf den Daten der Gruppe $i$ ist.
\end{block}
\begin{block}{Problem:}
In Anwendungen ist diese \textit{empirische} Varianz oft deutlich größer als die durch ein binomiales Regressionsmodell vorhergesagte Varianz $\frac{\hat{\pi}_{i}(1- \hat{\pi}_{i})}{n_{i}}$ mit $\hat{\pi}_{i}=h(x'_{i}\beta)$. \\
%\Longrightarrow Überdispersion
\end{block}

\begin{block}{Lösung:}
Einführung eines multiplikativen Überdispersionsparameters $\phi > 1$ in die Varianzformel, d.h. $Var(y_{i}|\textbf{x}_{i})=\phi \frac{\pi_{i}(1-\pi_{i})}{n_{i}}$

\end{block}

\tiny{(Quelle: Fahrmeir, Ludwig, et al. Regression)}
\end{frame}

\end{document}
